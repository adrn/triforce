%\documentclass{emulateapj}
%\documentclass[letterpaper,12pt,preprint]{aastex}
\documentclass[usenatbib]{mn2e}
\bibliographystyle{mn2e}


% packages
\usepackage{amssymb,amsmath,amsbsy}
\usepackage{booktabs}
\usepackage[caption=false]{subfig}
\usepackage{color}

% commands
\newcommand{\given}{\,|\,}
\newcommand{\dd}{\mathrm{d}}
\newcommand{\transpose}[1]{{#1}^{\mathsf{T}}}
\newcommand{\inverse}[1]{{#1}^{-1}}
\newcommand{\msun}{\mathrm{M}_\odot}

\newcommand{\project}[1]{\textsl{#1}}
\newcommand{\superfreq}{\project{SuperFreq}}

% TO DO
\usepackage{graphicx} 
\newcommand{\apwtodo}[1]{{\color{red} APW: (\MakeUppercase{#1})}}
\newcommand{\djdtodo}[1]{{\color{green} DJD: (\MakeUppercase{#1})}}

%Dan's Math Definitions
\newcommand{\scripty}[1]{\ensuremath{\mathcalligra{#1}}}
\def\sr{\scripty{r}}
\def\Mach{\mathcal{M}}
\def\bin{\rm{bin}}
\def\Mdot{\dot{M}}
\def\Msun{ M_{ \rm{\odot} } }

\def\xb{\bar{x}}
\def\yb{\bar{y}}








\begin{document}

\title{Cavity search: a parameter study of the importance of chaos for feeding black hole binaries}
\author[D. J. D'Orazio, Adrian M. Price-Whelan]{Daniel J. D'Orazio$^1$, Adrian M. Price-Whelan$^1$  
    \thanks{dorazio@astro.columbia.edu; adrn@astro.columbia.edu}\\
     $^1$Department of Astronomy, Columbia University, 550 West 120th Street, New York, NY 10027 
}

%\author{
%Daniel J. D'Orazio\altaffilmark{\colum},
%Adrian M. Price-Whelan\altaffilmark{\colum}
%}

% Affiliations
%\newcommand{\colum}{1}
% \newcommand{\adrn}{2}

%\altaffiltext{\colum}{Department of Astronomy,
   %                   Columbia University,
      %                550 W 120th St.,
         %             New York, NY 10027, USA}
% \altaffiltext{\adrn}{To whom correspondence should be addressed: adrn@astro.columbia.edu}

\maketitle


\begin{abstract}
%Danny boy
% Context
The interaction of a binary and a thin gasesous disk lies at the heart
of a number of important astrophysical phenomena. These include the
formation of planetary systems and the fate of massive black hole
binaries at the centers of galactic nuclei.
%The masses and orbital distributions of planetary systems are
%dictated by the interaction of a planet-star binary with a
%proto-planetery disk. The
%merger rates and electromagnetic signatures of massive black hole
%binares are influenced by gas which is torqued to the centers of
%galactic nuclei upon merger. Their fate is tied to the level
%of orbital evolution and mass feeding influenced by the coupled nature
%of binary+disk evolution. 
The salient features of binary+disk interactions are captured largley
by the gravitaitonal dynamics of a disk of particles in the plane of
the binary.
%Aims
Here we gain insight into binary+disk dynamics by studying orbits of
test particles in the binary plane via the restricted three body
problem.
%Methods
We investigate the resonant structure of orbits over a range of binary
mass ratios and orbital eccentricities relvant to astrophysical
systems.
% Results
We find some cool chaos crap related to Dan's other paper on CBD
transitions and also some dynamics stuff related to Adrian's work.
% Conclusions


\end{abstract}

%\keywords{ 
%stuff
%}

\section{Introduction}\label{sec:introduction}

\section{Methods}\label{sec:methods}

\subsection{Equations of Motions}
%\section{Equations of Motions}
The equations of motion for a test particle in the plane of an
eccentric binary are described by the elliptical, restricted three
body problem (ER3Bp). We write the ER3Bp as two coupled second order
ODE's in the non-uniformly rotating, isotropically pulsating, frame of
the eccentric binary ($\bar{x}$, $\bar{y}$),
\begin{equation}
\begin{array}{c}
\ddot{\xb} - 2\dot{\yb} = \frac{\partial{U}}{\partial{\xb}}  \left( 1 + e \cos{f} \right)^{-1} \nonumber \\ \nonumber \\ 
\ddot{\yb} + 2\dot{\xb} = \frac{\partial{U}}{\partial{\yb}}  \left( 1 + e \cos{f} \right)^{-1}
\label{Eqmotion}
\end{array}
\end{equation}
for binary with true anomaly $f$, eccentricity $e$, mean motion $\equiv 1$,
and $ \ \dot{} \ \equiv d/df$. Because the independent variable is
$f$, there is no need to solve for $f$ in terms of the time $t$ unless
converting the solution back to the non-rotating frame.

The ($\bar{x}$, $\bar{y}$) are dimensionless coordinates of the
massless third particle which are constructed by dividing the
dimensional coordinates by the time changing binary separation
\begin{equation}
\begin{array}{c}
\xb = \xb^* \frac{1 + e \cos{f}}{a ( 1-e^2) } \nonumber \\ \nonumber \\ 
\yb = \yb^* \frac{1 + e \cos{f}}{a ( 1-e^2) } 
\end{array}
\end{equation}
where $*$ denotes the dimensional variable.

The position of the primaries is found from dividing the dimensional
position over time by the binary position and is always fixed at the
same values as in the circular R3B, \textit{i.e.}
\begin{equation} \nonumber
\xb_2 = \xb^*_2 \frac{1+e \cos{f}}{a (1-e^2)} =   \frac{a}{1+q} \frac{(1-e^2)}{1+e \cos{f}}  \frac{1+e \cos{f}}{a (1-e^2)}  = \frac{1}{1+q}
\end{equation}
where $q = M_2/M_1$, $M_2 < M_1$.Similarly $\xb_p = -q/(1+q)$.

There is no longer a conserved quantity in the ER3B, but there is an
analogue to the Jacobi constant which is time dependent,
\begin{equation}
C_{eJ} = \frac{2 U}{1 + e \cos{f}} - \left( \dot{\xb}^2 + \dot{\yb}^2 \right) - 2 e \int^f_{f_0}{\frac{ U \sin{f} }{(1 + e \cos{f})^2 } \ df}
\end{equation}
and can be used to check the accuracy of the integrator (not yet
implemented).

\subsection{Set-up}
To set up initial velocity profiles recall that the independent
variable is no longer the time, but the true anomaly $f$. This can be
taken into account most simply by using the Virial theorem with the
pseudo potential of Eq. \ref{Eqmotion}. Assuming only azimuthal
initial velocities,
\begin{equation}
\begin{array}{c}
\bar{v_x} = -v_{\phi} \frac{\yb_0}{R} \quad \bar{v_x} = v_{\phi} \frac{\xb_0}{R} \\ \nonumber \\ \nonumber
v_{\phi} =   \left[ \sqrt{ \frac{\mu_1}{\bar{r}_1}  +  \frac{\mu_2}{\bar{r}_2} }  \right] \left( 1 + e \frac{\xb}{R} \right)^{-1}  - R\\ \nonumber \\ \nonumber
\bar{r}^2_1 =  (\xb + \mu_2)^2 + \yb^2  \qquad \bar{r}^2_2 =  (\xb - \mu_1)^2 + \yb^2  \\ \nonumber \\ \nonumber
R^2 = \xb^2 + \yb^2 \\ \nonumber \\ \nonumber
\mu_1 = \frac{1}{1+q} = 1-\mu_2  \qquad \mu_2 = \frac{q}{1+q}
\end{array}
\end{equation}
where we subtract the speed of the rotating frame from the azimuthal
velocity. Since the angular frequency of the rotating frame is unity
in our coordinates we need only subtract the particle specific
velocity $R$.

%If we were going to convert back into the time coordinate, we would need to use
%\begin{equation}
%df  = n \sqrt{1-e^2} \left( \frac{a}{r}\right)^2 dt = \frac{n (1 + e \cos{f})^2}{(1-e^2)^{3/2}}  dt
%\end{equation}
%where the last line substitutes the time dependent binary separation $r$. 

To convert back to time t, and unscaled distances we multiply the
coordinates by the time dependent binary separation and solve for
$t(f)$ in the usual manner.



Describe potential and things varied (mass ratio, viscosity,
eccentricity, what else?)\\ Lets start with mass ratio - look for
Linblad resonances, and see what happens when linear stability is
lost.

Orbits are integrated with ...

\subsection{Numerical determination of the fundamental frequencies}\label{sec:freqs}

Regular orbits in Hamiltonian systems may be represented in a special set of coordinates known as angle-action variables \citep[e.g.,][]{goldstein80}: in these coordinates, the position variables---the angles---increase linearly with time with rates set by a set of fundamental frequencies, and the momentum variables---the actions---are integrals of motion. More specifically, for a regular orbit represented in coordinates $(x_n, v_n)$, there exists a transformation $(x_n, v_n)\rightarrow(\theta_n, J_n)$ such that
\begin{align}
	\theta_n(t) &= \theta_n(0) + \Omega_n\,t\\
	J_n &= {\rm const.}
\end{align}
The $\Omega_n$ are known as the fundamental frequencies and there are $N$ such frequencies for non-resonant orbits in systems with $N$ degrees of freedom.

\apwtodo{Need to use a different symbol here for integer vector}
A resonant orbit is an orbit for which there exists at least one relation such that\footnote{The repeated indices imply summation.} $n_n\,\Omega_n = 0$ where $n_n$ is a vector of integers. \apwtodo{More theory-speakery here methinks...}

[How to numerically determine the frequencies using:]
\begin{equation}
	x(t) = \sum a_k\,e^{i \, \omega_k \, t}
\end{equation}

[How to detect chaos with this shite and measure rate of frequency diffusion]

\section{Conclusions}\label{sec:conclusions}

%\acknowledgements
\section{Acknowledgments}
APW is supported by a National Science Foundation Graduate Research Fellowship under Grant No.\ 11-44155. DJD is supported by a National Science Foundation Graduate Research Fellowship under Grant No. DGE1144155
This work was supported in part by the National Science Foundation under Grant No. PHYS-1066293.
This research made use of Astropy, a community-developed core Python package for Astronomy \citep{astropy13}.
This work additionally relied on Columbia University's \emph{Hotfoot} and \emph{Yeti} compute clusters, and we acknowledge the Columbia HPC support staff for assistance, especially Mr. Alex Bergier. 

%\bibliographystyle{apj}
%\bibliography{refs}

\end{document}
