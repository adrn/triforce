%\documentclass{emulateapj}
\documentclass[letterpaper,12pt,preprint]{aastex}

% packages
\usepackage{amssymb,amsmath,amsbsy}
\usepackage{booktabs}
\usepackage[caption=false]{subfig}

% commands
\newcommand{\given}{\,|\,}
\newcommand{\dd}{\mathrm{d}}
\newcommand{\transpose}[1]{{#1}^{\mathsf{T}}}
\newcommand{\inverse}[1]{{#1}^{-1}}
\newcommand{\msun}{\mathrm{M}_\odot}

\begin{document}

\title{ }
\author{
Daniel J. D'Orazio\altaffilmark{\colum},
Adrian M. Price-Whelan\altaffilmark{\colum}
}

% Affiliations
\newcommand{\colum}{1}
% \newcommand{\adrn}{2}

\altaffiltext{\colum}{Department of Astronomy,
                      Columbia University,
                      550 W 120th St.,
                      New York, NY 10027, USA}
% \altaffiltext{\adrn}{To whom correspondence should be addressed: adrn@astro.columbia.edu}

\begin{abstract}
Test.

% Context

% Aims

% Methods

% Results

% Conclusions

\end{abstract}

\keywords{ 
stuff
}

\section{Introduction}\label{sec:introduction}

\section{Conclusions}\label{sec:conclusions}

\acknowledgements
APW is supported by a National Science Foundation Graduate Research Fellowship under Grant No.\ 11-44155.
This work was supported in part by the National Science Foundation under Grant No. PHYS-1066293.
This research made use of Astropy, a community-developed core Python package for Astronomy \citep{astropy13}.
This work additionally relied on Columbia University's \emph{Hotfoot} and \emph{Yeti} compute clusters, and we acknowledge the Columbia HPC support staff for assistance. \\

%\bibliographystyle{apj}
%\bibliography{refs}

\end{document}
